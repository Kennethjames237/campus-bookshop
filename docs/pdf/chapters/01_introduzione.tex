% =============================================================================
% CHAPTER 1: INTRODUZIONE
% =============================================================================

\chapter{Introduzione}

\section{Il Team}

\begin{table}[H]
\centering
\begin{tabular}{lll}
\toprule
\textbf{Studente} & \textbf{Matricola} & \textbf{Ruolo Principale} \\
\midrule
Andrea Belli & 361054 & Database \\
Nna Minkousse Kenneth James & 366361 & Frontend \\
Samuele Premori & 361939 & Backend \\
\bottomrule
\end{tabular}
\caption{Composizione del Team T24}
\end{table}

\section{Quick Start}

\begin{enumerate}
    \item \textbf{Clonare il repository}
    
    \item \textbf{Configurare l'ambiente}
    \begin{lstlisting}[style=bashStyle]
cp .env.example .env
    \end{lstlisting}
    
    \item \textbf{Avviare i container}
    \begin{lstlisting}[style=bashStyle]
docker-compose up -d
    \end{lstlisting}
    
    \item \textbf{Accedere all'applicazione}
    \begin{itemize}
        \item Frontend: \url{http://localhost:8080}
        \item Backend API: \url{http://localhost:8081}
    \end{itemize}
\end{enumerate}

\section{Gestione Dati e Database}

\begin{itemize}
    \item \textbf{Rimuovere il Volume:}
    \begin{lstlisting}[style=bashStyle]
docker-compose down -v
    \end{lstlisting}
    
    \item \textbf{Avvio Build forzata Immagine:}
    \begin{lstlisting}[style=bashStyle]
docker-compose up -d --build
    \end{lstlisting}
\end{itemize}
