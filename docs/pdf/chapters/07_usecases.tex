% =============================================================================
% CHAPTER 7: USE CASES
% =============================================================================

\chapter{Use Cases}
\label{chap:usecases}

% =============================================================================
% UC01 - REGISTRAZIONE UTENTE
% =============================================================================
\section{UC01 - Registrazione Utente}

\subsection{Panoramica}

\textbf{Descrizione:} Consente a un visitatore di creare un nuovo account nel sistema, permettendogli successivamente di accedere alle funzionalit\`a riservate.

\begin{table}[H]
\centering
\begin{tabular}{ll}
\toprule
\textbf{Attori} & Visitatore \\
\textbf{Pre-condizioni} & Il visitatore non \`e autenticato e si trova nella pagina di registrazione \\
\textbf{Post-condizioni} & L'account \`e creato nel DB e il visitatore pu\`o fare login \\
\bottomrule
\end{tabular}
\end{table}

\begin{figure}[H]
    \centering
    \includesvg[width=0.6\textwidth]{uc01_registration/img/uc01.drawio}
    \caption{Use Case Diagram - UC01 Registrazione}
\end{figure}

\subsection{Flussi di Eventi}

\subsubsection{Flusso Principale}
\begin{enumerate}
    \item Il \textbf{Visitatore} accede alla pagina di registrazione.
    \item Il sistema mostra il modulo (Nome, Email, Password).
    \item Il \textbf{Visitatore} compila i campi e conferma.
    \item Il sistema verifica che l'email non sia gi\`a registrata.
    \item Il sistema valida il formato dei dati (es. password strong).
    \item Il sistema crea il nuovo account nel database.
    \item Il sistema reindirizza al login con messaggio di successo.
\end{enumerate}

\subsubsection{Flussi Alternativi}

\textbf{A1: Email gi\`a registrata}
\begin{enumerate}
    \item Il sistema rileva che l'email esiste.
    \item Il sistema mostra errore: ``Email gi\`a in uso''.
    \item Il flusso termina (l'utente deve riprovare).
\end{enumerate}

\textbf{A2: Dati non validi}
\begin{enumerate}
    \item Il sistema rileva formato errato.
    \item Il sistema evidenzia i campi in rosso.
    \item L'utente corregge e si torna al passo 3.
\end{enumerate}

\subsection{Activity Diagram}

\begin{figure}[H]
    \centering
    \includesvg[width=0.7\textwidth]{uc01_registration/img/uc01_flowchart.drawio}
    \caption{Activity Diagram - UC01 Registrazione}
\end{figure}

\subsection{Criteri di Accettazione e Testing}

\begin{itemize}
    \item La password deve avere almeno 8 caratteri.
    \item Se la registrazione ha successo, l'utente non viene loggato automaticamente ma va al login.
    \item La mail deve contenere ``@'' e un dominio valido.
\end{itemize}

\begin{table}[H]
\centering
\begin{tabular}{llll}
\toprule
\textbf{ID Test} & \textbf{Azione} & \textbf{Risultato Atteso} & \textbf{Valida} \\
\midrule
T01 & Invia form vuoto & Bordi rossi sui campi & $\checkmark$ \\
T02 & Invia mail esistente & Messaggio errore specifico & $\checkmark$ \\
T03 & Invia dati corretti & Redirect a /login & $\checkmark$ \\
\bottomrule
\end{tabular}
\caption{Piano di Test - UC01}
\end{table}

\subsection{Specifiche Tecniche}

\begin{figure}[H]
    \centering
    \includesvg[width=0.85\textwidth]{uc01_registration/img/uc01_sequence.drawio}
    \caption{Sequence Diagram - UC01}
\end{figure}

\begin{figure}[H]
    \centering
    \includesvg[width=0.85\textwidth]{uc01_registration/img/auth_class.drawio}
    \caption{Class Diagram Backend - Autenticazione}
\end{figure}

\begin{figure}[H]
    \centering
    \includesvg[width=0.7\textwidth]{uc01_registration/img/uc01_backend_flowchart.drawio}
    \caption{Backend Flowchart - UC01}
\end{figure}

% =============================================================================
% UC02 - LOGIN UTENTE
% =============================================================================
\section{UC02 - Login Utente}

\subsection{Panoramica}

\textbf{Descrizione:} Consente a un visitatore registrato di autenticarsi nel sistema per accedere alle funzionalit\`a riservate agli utenti (come la pubblicazione di annunci o l'acquisto).

\begin{table}[H]
\centering
\begin{tabular}{ll}
\toprule
\textbf{Attori} & Visitatore (Registrato) \\
\textbf{Pre-condizioni} & Il visitatore possiede un account e non \`e gi\`a autenticato \\
\textbf{Post-condizioni} & Il sistema crea una sessione/token e l'utente viene autenticato \\
\bottomrule
\end{tabular}
\end{table}

\begin{figure}[H]
    \centering
    \includesvg[width=0.6\textwidth]{uc02_login/img/uc02.drawio}
    \caption{Use Case Diagram - UC02 Login}
\end{figure}

\subsection{Flussi di Eventi}

\subsubsection{Flusso Principale}
\begin{enumerate}
    \item Il \textbf{Visitatore} accede alla pagina di login.
    \item Il sistema mostra il modulo (Email e Password).
    \item Il \textbf{Visitatore} inserisce le proprie credenziali e conferma.
    \item Il sistema valida il formato dei dati (lato Frontend).
    \item Il sistema verifica la corrispondenza delle credenziali nel database (lato Backend).
    \item Il sistema genera un token di sessione.
    \item Il sistema reindirizza l'utente alla Dashboard/Home con messaggio di successo.
\end{enumerate}

\subsubsection{Flussi Alternativi}

\textbf{A1: Credenziali errate o Account inesistente}
\begin{enumerate}
    \item Il sistema rileva che l'email non esiste o la password \`e errata.
    \item Il sistema mostra un messaggio di errore generico: ``Email o password non corretti''.
    \item Il flusso riprende dal punto 3.
\end{enumerate}

\begin{nota}
Si usa un messaggio generico per evitare che malintenzionati scoprano quali email sono registrate.
\end{nota}

\textbf{A2: Formato dati non valido}
\begin{enumerate}
    \item Il sistema rileva che la mail non \`e nel formato corretto o i campi sono vuoti.
    \item Il sistema blocca l'invio e segnala i campi da correggere.
\end{enumerate}

\subsection{Activity Diagram}

\begin{figure}[H]
    \centering
    \includesvg[width=0.7\textwidth]{uc02_login/img/uc02_flowchart.drawio}
    \caption{Activity Diagram - UC02 Login}
\end{figure}

\subsection{Criteri di Accettazione}

\begin{itemize}
    \item \textbf{Sicurezza:} Le password non devono mai apparire in chiaro nel modulo.
    \item \textbf{Sessione:} Al login effettuato, il sistema deve memorizzare lo stato di ``Loggato''.
    \item \textbf{Feedback:} In caso di errore, i campi non devono essere resettati per permettere la correzione veloce.
\end{itemize}

\subsection{Piano di Test Manuale}

\begin{table}[H]
\centering
\begin{tabular}{lp{5cm}p{5cm}l}
\toprule
\textbf{ID} & \textbf{Azione} & \textbf{Risultato Atteso} & \textbf{Valida} \\
\midrule
T01 & Inserire email non registrata & Messaggio di errore generico & $\checkmark$ \\
T02 & Inserire email corretta ma password errata & Messaggio di errore generico & $\checkmark$ \\
T03 & Lasciare il campo password vuoto & Tasto ``Login'' disabilitato o errore & $\checkmark$ \\
T04 & Inserire credenziali corrette & Reindirizzamento alla Home & $\checkmark$ \\
\bottomrule
\end{tabular}
\caption{Piano di Test - UC02}
\end{table}

\subsection{Design Tecnico}

\begin{figure}[H]
    \centering
    \includesvg[width=0.85\textwidth]{uc02_login/img/uc02_sequence.drawio}
    \caption{Sequence Diagram - UC02}
\end{figure}

\begin{figure}[H]
    \centering
    \includesvg[width=0.7\textwidth]{uc02_login/img/uc02_backend_flowchart.drawio}
    \caption{Backend Flowchart - UC02}
\end{figure}

\begin{figure}[H]
    \centering
    \includesvg[width=0.7\textwidth]{uc02_login/img/jwt_lifecycle.drawio}
    \caption{Ciclo di Vita JWT Token}
\end{figure}

% =============================================================================
% UC03 - VISUALIZZAZIONE LISTA LIBRI
% =============================================================================
\section{UC03 - Visualizzazione Lista Libri}

\subsection{Panoramica}

\textbf{Descrizione:} Consente a un Visitatore o a un Utente Autenticato di consultare la bacheca degli annunci disponibili. Se l'utente \`e loggato, il sistema nasconde automaticamente i suoi annunci.

\begin{table}[H]
\centering
\begin{tabular}{ll}
\toprule
\textbf{Attori} & Visitatore, Utente Autenticato, Database Annunci \\
\textbf{Pre-condizioni} & L'utente accede alla pagina ``Bacheca'' o ``Home'' \\
\textbf{Post-condizioni} & Il sistema mostra i libri filtrati disponibili per l'acquisto \\
\bottomrule
\end{tabular}
\end{table}

\begin{figure}[H]
    \centering
    \includesvg[width=0.7\textwidth]{uc03_listing/img/uc03.drawio}
    \caption{Use Case Diagram - UC03 Visualizzazione}
\end{figure}

\subsection{Flussi di Eventi}

\subsubsection{Flusso Principale}
\begin{enumerate}
    \item L'\textbf{Utente/Visitatore} accede alla bacheca dei libri.
    \item Il sistema (Presenter) richiede la lista degli annunci al Model (API).
    \item Il sistema (Backend) recupera i libri dal database.
    \item \textbf{Filtro Identit\`a:} Se l'utente \`e autenticato, il sistema esclude dalla lista gli annunci creati dall'utente stesso.
    \item Il sistema mostra la lista ordinata (es. dal pi\`u recente) con: Titolo, Prezzo, Foto e Autore.
    \item L'\textbf{Utente/Visitatore} visualizza i risultati.
\end{enumerate}

\subsubsection{Flussi Alternativi}

\textbf{A1: Nessun libro disponibile}
\begin{enumerate}
    \item Il database non restituisce risultati.
    \item Il sistema mostra un messaggio: ``Al momento non ci sono libri disponibili. Torna pi\`u tardi!''.
\end{enumerate}

\textbf{A2: Errore caricamento immagini}
\begin{enumerate}
    \item Se la foto di un annuncio non \`e reperibile, il sistema mostra un'immagine segnaposto (placeholder) di default.
\end{enumerate}

\subsection{Activity Diagram}

\begin{figure}[H]
    \centering
    \includesvg[width=0.7\textwidth]{uc03_listing/img/uc03_flowchart.drawio}
    \caption{Activity Diagram - UC03 Visualizzazione}
\end{figure}

\subsection{Criteri di Accettazione}

\begin{itemize}
    \item La lista deve essere caricata in modo asincrono senza ricaricare la pagina.
    \item L'utente non deve visualizzare i propri annunci nella lista.
    \item Ogni annuncio deve mostrare chiaramente il prezzo.
\end{itemize}

\subsection{Piano di Test Manuale}

\begin{table}[H]
\centering
\begin{tabular}{lp{4cm}p{5cm}l}
\toprule
\textbf{ID} & \textbf{Azione} & \textbf{Risultato Atteso} & \textbf{Valida} \\
\midrule
T01 & Accesso da Visitatore & Visualizzazione di tutti i libri nel DB & $\checkmark$ \\
T02 & Accesso da Utente Loggato & I propri libri sono nascosti & $\checkmark$ \\
T03 & Database vuoto & Messaggio ``Nessun libro disponibile'' & $\checkmark$ \\
\bottomrule
\end{tabular}
\caption{Piano di Test - UC03}
\end{table}

\subsection{Design Tecnico}

\begin{figure}[H]
    \centering
    \includesvg[width=0.85\textwidth]{uc03_listing/img/uc03_sequence.drawio}
    \caption{Sequence Diagram - UC03}
\end{figure}

\begin{figure}[H]
    \centering
    \includesvg[width=0.7\textwidth]{uc03_listing/img/uc03_backend_flowchart.drawio}
    \caption{Backend Flowchart - UC03}
\end{figure}

\begin{figure}[H]
    \centering
    \includesvg[width=0.85\textwidth]{uc03_listing/img/books_class.drawio}
    \caption{Class Diagram - Catalogo Libri}
\end{figure}

% =============================================================================
% UC04 - PUBBLICAZIONE ANNUNCIO
% =============================================================================
\section{UC04 - Pubblicazione Annuncio}

\subsection{Panoramica}

\textbf{Descrizione:} Consente a un Utente Autenticato di inserire un nuovo annuncio di vendita per un libro, specificando dettagli accademici e caricando una fotografia.

\begin{table}[H]
\centering
\begin{tabular}{ll}
\toprule
\textbf{Attori} & Utente Autenticato (Venditore) \\
\textbf{Pre-condizioni} & L'utente \`e loggato e si trova nel form di creazione annuncio \\
\textbf{Post-condizioni} & L'annuncio \`e salvato nel DB e l'immagine \`e memorizzata sul server \\
\bottomrule
\end{tabular}
\end{table}

\begin{figure}[H]
    \centering
    \includesvg[width=0.6\textwidth]{uc04_adding/img/uc04.drawio}
    \caption{Use Case Diagram - UC04 Pubblicazione}
\end{figure}

\subsection{Flussi di Eventi}

\subsubsection{Flusso Principale}
\begin{enumerate}
    \item Il \textbf{Venditore} clicca su ``Pubblica Annuncio''.
    \item Il sistema mostra il modulo di inserimento: Titolo, Autore, ISBN, Prezzo, Corso, Docente e Caricamento Foto.
    \item Il \textbf{Venditore} compila i campi e seleziona un'immagine dal dispositivo.
    \item Il sistema (Frontend) valida la presenza dei campi obbligatori e il formato numerico del prezzo.
    \item Il sistema (Backend) riceve i dati, salva l'immagine nel filesystem e crea il record nel database.
    \item Il sistema reindirizza l'utente alla propria area personale o alla bacheca.
\end{enumerate}

\subsubsection{Flussi Alternativi}

\textbf{A1: Dati mancanti o Formato errato}
\begin{enumerate}
    \item Il sistema evidenzia i campi non validi (es. ISBN non numerico o prezzo negativo).
    \item Il pulsante di invio viene disabilitato finch\'e i dati non sono corretti.
\end{enumerate}

\textbf{A2: Errore Caricamento Immagine}
\begin{enumerate}
    \item Il file caricato non \`e un'immagine o supera la dimensione massima.
    \item Il sistema mostra l'errore: ``Formato file non supportato o file troppo grande''.
\end{enumerate}

\subsection{Activity Diagram}

\begin{figure}[H]
    \centering
    \includesvg[width=0.7\textwidth]{uc04_adding/img/uc04_flowchart.drawio}
    \caption{Activity Diagram - UC04 Pubblicazione}
\end{figure}

\subsection{Criteri di Accettazione}

\begin{itemize}
    \item Il campo ISBN deve accettare solo numeri (10 o 13 cifre).
    \item Il prezzo deve essere obbligatoriamente un numero positivo.
    \item L'utente deve poter caricare una foto.
    \item L'annuncio deve essere collegato all'ID dell'utente che lo ha creato.
\end{itemize}

\subsection{Piano di Test Manuale}

\begin{table}[H]
\centering
\begin{tabular}{lp{4cm}p{5cm}l}
\toprule
\textbf{ID} & \textbf{Azione} & \textbf{Risultato Atteso} & \textbf{Valida} \\
\midrule
T01 & Tentativo di invio con campi vuoti & Errore ``Campi obbligatori mancanti'' & $\checkmark$ \\
T02 & Inserimento prezzo negativo & Errore di validazione sul campo prezzo & $\checkmark$ \\
T03 & Caricamento file non immagine & Blocco del caricamento e avviso & $\checkmark$ \\
T04 & Invio corretto dei dati & Successo e comparsa in bacheca & $\checkmark$ \\
\bottomrule
\end{tabular}
\caption{Piano di Test - UC04}
\end{table}

\subsection{Design Tecnico}

\begin{figure}[H]
    \centering
    \includesvg[width=0.85\textwidth]{uc04_adding/img/uc04_sequence.drawio}
    \caption{Sequence Diagram - UC04}
\end{figure}

\begin{figure}[H]
    \centering
    \includesvg[height=0.6\textheight]{uc04_adding/img/uc04_backend_flowchart.drawio}
    \caption{Backend Flowchart - UC04}
\end{figure}

\begin{figure}[H]
    \centering
    \includesvg[height=0.55\textheight]{uc04_adding/img/image_upload_flow.drawio}
    \caption{Flusso Upload Immagine}
\end{figure}

% =============================================================================
% UC05 - ACQUISTO DIRETTO
% =============================================================================
\section{UC05 - Acquisto Diretto}

\subsection{Panoramica}

\textbf{Descrizione:} Consente a un Utente Autenticato di acquisire un libro. L'operazione \`e puramente logica (non c'\`e transazione monetaria reale): il libro sparisce dalla bacheca e viene spostato negli storici dei due utenti coinvolti.

\begin{table}[H]
\centering
\begin{tabular}{ll}
\toprule
\textbf{Attori} & Utente Autenticato (Acquirente) \\
\textbf{Pre-condizioni} & L'acquirente \`e loggato e il libro ha stato ``Disponibile'' \\
\textbf{Post-condizioni} & Libro impostato come ``Venduto'', rimosso dalla bacheca e aggiunto agli storici \\
\bottomrule
\end{tabular}
\end{table}

\begin{figure}[H]
    \centering
    \includesvg[width=0.7\textwidth]{uc05_buy/img/uc05.drawio}
    \caption{Use Case Diagram - UC05 Acquisto}
\end{figure}

\subsection{Flussi di Eventi}

\subsubsection{Flusso Principale}
\begin{enumerate}
    \item L'\textbf{Acquirente} visualizza i dettagli di un libro nella bacheca.
    \item L'\textbf{Acquirente} clicca sul pulsante ``Acquista''.
    \item Il sistema chiede conferma dell'operazione.
    \item L'\textbf{Acquirente} conferma.
    \item Il sistema (Backend) verifica che il libro sia ancora nello stato ``Disponibile''.
    \item Il sistema crea un record nella tabella \code{transactions}.
    \item Il sistema aggiorna lo stato del libro in \code{status = 'sold'}.
    \item Il sistema mostra un messaggio di successo e fornisce i contatti del venditore.
\end{enumerate}

\subsubsection{Flussi Alternativi}

\textbf{A1: Libro gi\`a venduto}
\begin{enumerate}
    \item Il sistema rileva che il libro \`e appena stato acquistato da un altro utente.
    \item Il sistema mostra l'errore: ``Spiacenti, il libro \`e stato appena venduto''.
    \item L'utente viene reindirizzato alla bacheca aggiornata.
\end{enumerate}

\textbf{A2: Tentativo di auto-acquisto}
\begin{enumerate}
    \item Il sistema rileva che l'acquirente \`e anche il venditore del libro.
    \item Il pulsante ``Acquista'' \`e disabilitato o restituisce un errore di logica.
\end{enumerate}

\subsection{Activity Diagram}

\begin{figure}[H]
    \centering
    \includesvg[height=0.55\textheight]{uc05_buy/img/uc05_activity.drawio}
    \caption{Activity Diagram - UC05 Acquisto}
\end{figure}

\subsection{Criteri di Accettazione}

\begin{itemize}
    \item Un libro venduto non deve comparire nei risultati di ricerca.
    \item L'acquisto deve generare una voce nello storico ``Miei Acquisti'' dell'acquirente.
    \item L'acquisto deve generare una voce nello storico ``Libri Venduti'' del venditore.
    \item L'operazione deve essere atomica (se fallisce il salvataggio dello storico, il libro non deve risultare venduto).
\end{itemize}

\subsection{Piano di Test Manuale}

\begin{table}[H]
\centering
\begin{tabular}{lp{4cm}p{5cm}}
\toprule
\textbf{ID} & \textbf{Azione} & \textbf{Risultato Atteso} \\
\midrule
T01 & Cliccare ``Acquista'' e poi ``Annulla'' & Nessuna modifica, libro resta disponibile \\
T02 & Confermare l'acquisto & Successo, libro sparito dalla bacheca \\
T03 & Verificare ``Storico Acquisti'' & Il libro appena comprato appare in cima \\
T04 & Accesso diretto a libro gi\`a venduto & ``Libro non disponibile'' \\
\bottomrule
\end{tabular}
\caption{Piano di Test - UC05}
\end{table}

\subsection{Design Tecnico}

\begin{figure}[H]
    \centering
    \includesvg[width=0.85\textwidth]{uc05_buy/img/uc05_sequence.drawio}
    \caption{Sequence Diagram - UC05}
\end{figure}

\begin{figure}[H]
    \centering
    \includesvg[height=0.75\textheight]{uc05_buy/img/uc05_backend_flowchart.drawio}
    \caption{Backend Flowchart - UC05}
\end{figure}

\textbf{Flusso di validazione:}
\begin{enumerate}
    \item \textbf{Autenticazione JWT:} Verifica che il token sia valido e non scaduto.
    \item \textbf{Validazione Input:} Controllo presenza e formato del \code{bookId}.
    \item \textbf{Esistenza Libro:} Verifica che il libro esista nel database.
    \item \textbf{Disponibilit\`a:} Controllo che \code{book.available == true}.
    \item \textbf{Anti Auto-Acquisto:} Verifica che \code{buyer\_id != seller\_id}.
\end{enumerate}

\textbf{Operazione Atomica:}
La funzione \code{placeOrder()} esegue in una singola transazione database:
\begin{itemize}
    \item Creazione record nella tabella \code{transactions}.
    \item Aggiornamento del campo \code{available = false} sul libro.
\end{itemize}

% =============================================================================
% UC06 - RICERCA AVANZATA
% =============================================================================
\section{UC06 - Ricerca Avanzata}

\subsection{Panoramica}

\textbf{Descrizione:} Consente a un Utente o Visitatore di cercare libri nella bacheca filtrando per ISBN, corso o docente. La ricerca viene eseguita lato Frontend sui dati gi\`a caricati.

\begin{table}[H]
\centering
\begin{tabular}{ll}
\toprule
\textbf{Attori} & Visitatore, Utente Autenticato \\
\textbf{Pre-condizioni} & La bacheca dei libri \`e stata caricata \\
\textbf{Post-condizioni} & Vengono visualizzati solo i libri che corrispondono ai criteri \\
\bottomrule
\end{tabular}
\end{table}

\begin{figure}[H]
    \centering
    \includesvg[width=0.6\textwidth]{uc06_search/uc06}
    \caption{Use Case Diagram - UC06 Ricerca}
\end{figure}

\subsection{Flussi di Eventi}

\subsubsection{Flusso Principale}
\begin{enumerate}
    \item L'\textbf{Utente/Visitatore} visualizza la bacheca con tutti i libri disponibili.
    \item L'\textbf{Utente/Visitatore} inserisce uno o pi\`u criteri di ricerca (ISBN, corso, docente).
    \item Il sistema (Frontend) filtra i risultati in tempo reale.
    \item Il sistema mostra solo i libri che corrispondono ai criteri inseriti.
\end{enumerate}

\subsubsection{Flussi Alternativi}

\textbf{A1: Nessun risultato trovato}
\begin{enumerate}
    \item Nessun libro corrisponde ai criteri di ricerca.
    \item Il sistema mostra un messaggio: ``Nessun libro trovato per i criteri selezionati''.
\end{enumerate}

\textbf{A2: Ricerca parziale}
\begin{enumerate}
    \item L'utente inserisce solo parte di un criterio (es. alcune cifre dell'ISBN).
    \item Il sistema mostra tutti i libri che contengono la stringa parziale.
\end{enumerate}

\textbf{A3: Reset ricerca}
\begin{enumerate}
    \item L'utente svuota i campi di ricerca.
    \item Il sistema mostra nuovamente tutti i libri disponibili.
\end{enumerate}

\subsection{Criteri di Accettazione}

\begin{itemize}
    \item La ricerca per ISBN deve restituire corrispondenze esatte o parziali.
    \item La ricerca per corso deve essere case-insensitive.
    \item La ricerca per docente deve essere case-insensitive.
    \item \`E possibile combinare pi\`u criteri di ricerca contemporaneamente.
    \item La ricerca deve funzionare sia per Visitatori che per Utenti autenticati.
    \item Svuotando i campi di ricerca devono ricomparire tutti i libri.
\end{itemize}

\subsection{Piano di Test Manuale}

\begin{table}[H]
\centering
\begin{tabular}{lp{4cm}p{5cm}l}
\toprule
\textbf{ID} & \textbf{Azione} & \textbf{Risultato Atteso} & \textbf{Valida} \\
\midrule
T01 & Inserire un ISBN completo & Solo il libro con quell'ISBN & $\checkmark$ \\
T02 & Inserire un ISBN parziale & Libri il cui ISBN contiene quella sequenza & $\checkmark$ \\
T03 & Inserire un nome corso esistente & Tutti i libri associati a quel corso & $\checkmark$ \\
T04 & Inserire un nome docente & Tutti i libri associati a quel docente & $\checkmark$ \\
T05 & Criteri senza corrispondenze & Messaggio ``Nessun libro trovato'' & $\checkmark$ \\
T06 & Combinare ISBN + Corso & Solo libri che soddisfano entrambi & $\checkmark$ \\
T07 & Svuotare tutti i campi & Tornano visibili tutti i libri & $\checkmark$ \\
\bottomrule
\end{tabular}
\caption{Piano di Test - UC06}
\end{table}

\subsection{Design Tecnico}

\begin{figure}[H]
    \centering
    \includesvg[width=0.85\textwidth]{uc06_search/uc06_sequence}
    \caption{Sequence Diagram - UC06}
\end{figure}

\begin{figure}[H]
    \centering
    \includesvg[width=0.7\textwidth]{uc06_search/uc06_class.drawio}
    \caption{Class Diagram - Strategy Pattern per i Filtri}
\end{figure}

% =============================================================================
% UC07 - MESSAGGISTICA TRA UTENTI
% =============================================================================
\section{UC07 - Messaggistica tra Utenti}

\subsection{Panoramica}

\textbf{Descrizione:} Consente a un Utente Autenticato di inviare e ricevere messaggi diretti da altri utenti, per accordarsi su prezzo, pagamento e consegna dei libri.

\begin{table}[H]
\centering
\begin{tabular}{ll}
\toprule
\textbf{Attori} & Utente Autenticato (Mittente), Utente Autenticato (Destinatario) \\
\textbf{Pre-condizioni} & Entrambi gli utenti sono registrati. Il mittente \`e autenticato \\
\textbf{Post-condizioni} & Il messaggio \`e salvato nel DB e visibile al destinatario \\
\bottomrule
\end{tabular}
\end{table}

\begin{figure}[H]
    \centering
    \includesvg[width=0.7\textwidth]{uc07_messaging/img/uc07.drawio}
    \caption{Use Case Diagram - UC07 Messaggistica}
\end{figure}

\subsection{Flussi di Eventi}

\subsubsection{Flusso Principale}
\begin{enumerate}
    \item L'\textbf{Utente} accede alla sezione messaggi o clicca ``Contatta venditore'' da un annuncio.
    \item Il sistema mostra la lista delle conversazioni esistenti o apre una nuova conversazione.
    \item L'\textbf{Utente} scrive un messaggio nel campo di testo.
    \item L'\textbf{Utente} clicca su ``Invia''.
    \item Il sistema (Backend) salva il messaggio nel database con mittente, destinatario e timestamp.
    \item Il sistema aggiorna la conversazione mostrando il nuovo messaggio.
\end{enumerate}

\subsubsection{Flussi Alternativi}

\textbf{A1: Messaggio vuoto}
\begin{enumerate}
    \item L'utente tenta di inviare un messaggio senza contenuto.
    \item Il sistema disabilita il pulsante ``Invia'' o mostra un errore.
\end{enumerate}

\textbf{A2: Destinatario non trovato}
\begin{enumerate}
    \item Il sistema non trova l'utente destinatario (es. account eliminato).
    \item Il sistema mostra l'errore: ``Impossibile inviare il messaggio. Utente non trovato''.
\end{enumerate}

\textbf{A3: Visualizzazione nuovi messaggi}
\begin{enumerate}
    \item L'utente accede alla sezione messaggi.
    \item Il sistema evidenzia le conversazioni con messaggi non letti.
\end{enumerate}

\subsection{Activity Diagram}

\begin{figure}[H]
    \centering
    \includesvg[height=0.55\textheight]{uc07_messaging/img/uc07_flowchart.drawio}
    \caption{Activity Diagram - UC07 Messaggistica}
\end{figure}

\subsection{Criteri di Accettazione}

\begin{itemize}
    \item Solo gli utenti autenticati possono inviare e ricevere messaggi.
    \item Ogni messaggio deve essere associato a un mittente e un destinatario.
    \item I messaggi devono essere ordinati cronologicamente nella conversazione.
    \item L'utente deve poter visualizzare lo storico dei messaggi con un altro utente.
    \item I messaggi non letti devono essere evidenziati o contrassegnati.
    \item Non \`e possibile inviare messaggi a se stessi.
\end{itemize}

\subsection{Piano di Test Manuale}

\begin{table}[H]
\centering
\begin{tabular}{lp{4cm}p{6cm}}
\toprule
\textbf{ID} & \textbf{Azione} & \textbf{Risultato Atteso} \\
\midrule
T01 & Inviare un messaggio a un altro utente & Il messaggio appare nella conversazione di entrambi \\
T02 & Tentativo di invio messaggio vuoto & Pulsante ``Invia'' disabilitato o errore \\
T03 & Accedere con messaggi non letti & I messaggi non letti sono evidenziati \\
T04 & Visualizzare lo storico di una conversazione & Tutti i messaggi in ordine cronologico \\
T05 & Tentativo da utente non autenticato & Redirect alla pagina di login \\
T06 & Contattare venditore da un annuncio & Si apre la conversazione con il venditore \\
\bottomrule
\end{tabular}
\caption{Piano di Test - UC07}
\end{table}

\subsection{Design Tecnico}

\begin{figure}[H]
    \centering
    \includesvg[width=0.85\textwidth]{uc07_messaging/img/uc07_sequence.drawio}
    \caption{Sequence Diagram - UC07}
\end{figure}

\begin{figure}[H]
    \centering
    \includesvg[height=0.7\textheight]{uc07_messaging/img/uc07_backend_flowchart.drawio}
    \caption{Backend Flowchart - UC07}
\end{figure}

\textbf{Validazioni comuni a tutti gli endpoint:}
\begin{enumerate}
    \item \textbf{Autenticazione JWT:} Tutti gli endpoint richiedono un token JWT valido.
    \item \textbf{Estrazione senderId:} Il mittente viene sempre estratto dal token, mai dal body della richiesta.
\end{enumerate}

\textbf{Validazioni specifiche per POST /messages:}
\begin{enumerate}
    \item \textbf{Contenuto non vuoto:} Il messaggio deve avere contenuto.
    \item \textbf{Anti auto-invio:} Non \`e possibile inviare messaggi a se stessi (\code{senderId != receiverId}).
    \item \textbf{Destinatario esistente:} Verifica che l'utente destinatario esista nel sistema.
\end{enumerate}
