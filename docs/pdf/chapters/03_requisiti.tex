% =============================================================================
% CHAPTER 3: ANALISI DEI REQUISITI
% =============================================================================

\chapter{Analisi dei Requisiti}

\textbf{Scopo del Progetto:} Applicazione Web per la compravendita di libri universitari.

\begin{itemize}
    \item Non verr\`a implementato un sistema di pagamento.
    \item Non verr\`a implementato un meccanismo di password recovery.
    \item La gestione per accordarsi sul prezzo, su un diverso pagamento e sulla consegna del libro, sono attualmente lasciate alla gestione tra utenti.
\end{itemize}

\section{Requisiti Non Funzionali}

\begin{itemize}
    \item \textbf{RNF01:} Implementazione dell'architettura attraverso il pattern Model-View-Presenter (MVP).
    \item \textbf{RNF02:} Backend implementato in PHP puro.
    \item \textbf{RNF03:} Frontend implementato in HTML, CSS e Javascript.
    \item \textbf{RNF04:} Il frontend comunica con il backend solo tramite API RESTful CRUD.
    \item \textbf{RNF05:} Il Backend non deve generare codice HTML.
    \item \textbf{RNF06:} Le password non devono essere salvate in chiaro.
\end{itemize}

\section{Requisiti Funzionali}

\begin{itemize}
    \item \textbf{RF01:} Login e Registrazione degli utenti.
    \item \textbf{RF02:} Visualizzazione lista libri.
    \item \textbf{RF03:} Caricamento annuncio.
    \item \textbf{RF04:} Acquisto diretto.
    \item \textbf{RF05:} Ricerca avanzata per ISBN, corso, docente.
    \item \textbf{RF06:} Messaggistica tra utenti.
\end{itemize}

\section{Attori di Sistema Principali}

\subsection{Visitatore}
Utente non autenticato, pu\`o:
\begin{enumerate}
    \item Visualizzare tutti gli annunci.
    \item Effettuare il login.
    \item Effettuare la registrazione.
\end{enumerate}

\subsection{Utente}
Utente autenticato, pu\`o:
\begin{enumerate}
    \item Visualizzare annunci pubblicati da altri.
    \item Aggiungere, modificare e rimuovere i propri annunci.
    \item Visualizzare lo storico dei libri venduti.
    \item Visualizzare lo storico degli acquisti.
    \item Effettuare il logout.
\end{enumerate}

\section{User Stories}

\subsection{[US01] Registrazione Utente}
\begin{quote}
\textbf{Come} Visitatore,\\
\textbf{Voglio} registrarmi,\\
\textbf{Affinch\'e} il sistema mi permetta di accedere.
\end{quote}
\textbf{Analisi:} UC01 (Capitolo \ref{chap:usecases})

\subsection{[US02] Login Utente}
\begin{quote}
\textbf{Come} Visitatore,\\
\textbf{Voglio} effettuare il login,\\
\textbf{Affinch\'e} il sistema mi permetta di compiere operazioni.
\end{quote}
\textbf{Analisi:} UC02 (Capitolo \ref{chap:usecases})

\subsection{[US03] Visualizzazione}
\begin{quote}
\textbf{Come} Utente o Visitatore,\\
\textbf{Voglio} vedere la lista dei libri (esclusi i miei),\\
\textbf{Affinch\'e} possa trovare testi da acquistare.
\end{quote}
\textbf{Analisi:} UC03 (Capitolo \ref{chap:usecases})

\subsection{[US04] Pubblicazione Annuncio}
\begin{quote}
\textbf{Come} Utente,\\
\textbf{Voglio} caricare un libro (Titolo, Prezzo, Foto, ISBN, Docente, Corso),\\
\textbf{Affinch\'e} sia visibile e acquistabile dagli altri utenti.
\end{quote}
\textbf{Analisi:} UC04 (Capitolo \ref{chap:usecases})

\subsection{[US05] Acquisto Diretto}
\begin{quote}
\textbf{Come} Utente,\\
\textbf{Voglio} poter acquistare un libro,\\
\textbf{Affinch\'e} il libro vada nello storico dei miei acquisiti, nello storico annunci del venditore e sparisca dalla bacheca.
\end{quote}
\textbf{Analisi:} UC05 (Capitolo \ref{chap:usecases})

\subsection{[US06] Ricerca Avanzata}
\begin{quote}
\textbf{Come} Utente o Visitatore,\\
\textbf{Voglio} poter cercare libri per ISBN, corso o docente,\\
\textbf{Affinch\'e} possa trovare rapidamente i testi di mio interesse.
\end{quote}
\textbf{Analisi:} UC06 (Capitolo \ref{chap:usecases})

\subsection{[US07] Messaggistica tra Utenti}
\begin{quote}
\textbf{Come} Utente,\\
\textbf{Voglio} poter inviare e ricevere messaggi da altri utenti,\\
\textbf{Affinch\'e} possa accordarmi su prezzo e consegna dei libri.
\end{quote}
\textbf{Analisi:} UC07 (Capitolo \ref{chap:usecases})

\section{Matrice di Tracciabilit\`a}

\begin{table}[H]
\centering
\begin{tabular}{lcccccc}
\toprule
 & \textbf{RF01} & \textbf{RF02} & \textbf{RF03} & \textbf{RF04} & \textbf{RF05} & \textbf{RF06} \\
\midrule
UC01 & X &   &   &   &   &   \\
UC02 & X &   &   &   &   &   \\
UC03 &   & X &   &   &   &   \\
UC04 &   &   & X &   &   &   \\
UC05 &   &   &   & X &   &   \\
UC06 &   &   &   &   & X &   \\
UC07 &   &   &   &   &   & X \\
\bottomrule
\end{tabular}
\caption{Matrice di Tracciabilit\`a Use Cases - Requisiti Funzionali}
\end{table}
