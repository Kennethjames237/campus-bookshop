% =============================================================================
% CHAPTER 5: CONTRATTI DI INTERFACCIA
% =============================================================================

\chapter{Contratti di Interfaccia}

\section{Specifiche API}

In questa sezione vengono definiti gli endpoint del sistema. Tutte le richieste e le risposte utilizzano il formato \textbf{JSON}. Per le operazioni di scrittura e protezione dei dati, il sistema adotta lo standard \textbf{JWT} (JSON Web Token).

\begin{nota}
\textbf{Autenticazione:} Gli endpoint che richiedono autorizzazione devono includere l'header:\\
\code{Authorization: Bearer <token\_jwt>}
\end{nota}

\begin{nota}
\textbf{Risposta Errore Autorizzazione:} Se il token \`e mancante, invalido o scaduto, gli endpoint protetti restituiscono:
\begin{lstlisting}[language=json,style=jsonStyle]
{
  "status": "error",
  "message": "Unauthorized"
}
\end{lstlisting}
\end{nota}

%------------------------------------------------------------------------------
\subsection{POST /register}

Registrazione di un nuovo utente.

\textbf{Request Body:}
\begin{lstlisting}[language=json,style=jsonStyle]
{
  "email": "utente@esempio.it",
  "username": "mario_rossi",
  "password": "password"
}
\end{lstlisting}

\textbf{Response (Success):}
\begin{lstlisting}[language=json,style=jsonStyle]
{
  "status": "success",
  "message": "User registered successfully"
}
\end{lstlisting}

\textbf{Response (Error - Input non valido):}
\begin{lstlisting}[language=json,style=jsonStyle]
{
  "status": "error",
  "message": "Invalid input"
}
\end{lstlisting}

%------------------------------------------------------------------------------
\subsection{POST /login}

Autenticazione e rilascio del token.

\textbf{Request Body:}
\begin{lstlisting}[language=json,style=jsonStyle]
{
  "email": "utente@esempio.it",
  "password": "password"
}
\end{lstlisting}

\textbf{Response (Success):}
\begin{lstlisting}[language=json,style=jsonStyle]
{
  "status": "success",
  "message": "Login successful",
  "token": "eyJhbGciOiJIUzI1NiIsInR5cCI6IkpXVCJ9..."
}
\end{lstlisting}

\textbf{Response (Error - Input non valido):}
\begin{lstlisting}[language=json,style=jsonStyle]
{
  "status": "error",
  "message": "Invalid input"
}
\end{lstlisting}

\textbf{Response (Error - Credenziali errate):}
\begin{lstlisting}[language=json,style=jsonStyle]
{
  "status": "error",
  "message": "Invalid credentials"
}
\end{lstlisting}

\begin{nota}
\textbf{Sicurezza:} Il messaggio ``Invalid credentials'' \`e volutamente generico per prevenire attacchi di enumerazione utenti. Non viene mai rivelato se l'email \`e registrata o meno.
\end{nota}

%------------------------------------------------------------------------------
\subsection{GET /books}

Recupero della lista di tutti i libri disponibili.

\textbf{Auth:} Opzionale (JWT)

\begin{nota}
\textbf{Filtro Identit\`a:} Se l'utente \`e autenticato (JWT valido), il sistema esclude automaticamente i libri pubblicati dall'utente stesso dalla lista restituita.
\end{nota}

\textbf{Response (Success):}
\begin{lstlisting}[language=json,style=jsonStyle]
{
  "status": "success",
  "data": [
    {
      "id": 1,
      "name": "Ingegneria del Software",
      "author": "Ian Sommerville",
      "isbn": "978-8871926284",
      "imagePath": "data:image/jpeg;base64,/9j/4AAQ...",
      "teacher": "Prof. Bagnara",
      "course": "Informatica",
      "price": 35.00,
      "sellerId": 10,
      "sellerUsername": "username",
      "available": true
    }
  ]
}
\end{lstlisting}

\begin{nota}
\textbf{Immagine:} Il campo \code{imagePath} contiene l'immagine codificata in formato base64 data URI, pronta per essere utilizzata direttamente come attributo \code{src} di un tag \code{<img>}. Se il libro non ha un'immagine associata, il campo sar\`a una stringa vuota.
\end{nota}

\textbf{Response (Error - Errore server):}
\begin{lstlisting}[language=json,style=jsonStyle]
{
  "status": "error",
  "message": "Server error"
}
\end{lstlisting}

%------------------------------------------------------------------------------
\subsection{GET /my-books}

Recupera la lista di tutti i libri messi in vendita dall'utente correntemente loggato.

\textbf{Auth:} Obbligatoria (JWT)

\textbf{Response (Success):}
\begin{lstlisting}[language=json,style=jsonStyle]
{
  "status": "success",
  "data": [
    {
      "id": 5,
      "name": "Sistemi Operativi",
      "author": "Silberschatz",
      "isbn": "978-1118063330",
      "imagePath": "data:image/jpeg;base64,...",
      "teacher": "Prof. Veltri",
      "course": "Informatica",
      "price": 40.00,
      "sellerId": 10,
      "sellerUsername": "username",
      "available": true
    }
  ]
}
\end{lstlisting}

%------------------------------------------------------------------------------
\subsection{POST /books}

Inserimento di un nuovo libro nel catalogo.

\textbf{Auth:} JWT Required

\textbf{Request Body:}
\begin{lstlisting}[language=json,style=jsonStyle]
{
  "name": "Sistemi Operativi",
  "author": "Silberschatz",
  "isbn": "978-1122334455",
  "image": "data:image/jpeg;base64,/9j/4AAQ...",
  "teacher": "Prof. Bagnara",
  "course": "Informatica",
  "price": 28.00
}
\end{lstlisting}

\begin{nota}
Il campo \code{sellerId} viene estratto automaticamente dal JWT (claim \code{sub}) e non deve essere specificato nel body.
\end{nota}

\begin{nota}
\textbf{Upload Immagine:} Il campo \code{image} \`e opzionale e accetta dati in formato base64 (con o senza prefisso data URI). Formati supportati: JPEG, PNG, WebP. Dimensione massima: 5MB.
\end{nota}

\textbf{Response (Success):}
\begin{lstlisting}[language=json,style=jsonStyle]
{
  "status": "success",
  "id": 105,
  "message": "Libro messo in vendita con successo"
}
\end{lstlisting}

\textbf{Response (Error - Input non valido):}
\begin{lstlisting}[language=json,style=jsonStyle]
{
  "status": "error",
  "message": "Invalid input"
}
\end{lstlisting}

\textbf{Response (Error - Formato immagine non valido):}
\begin{lstlisting}[language=json,style=jsonStyle]
{
  "status": "error",
  "message": "Invalid file format"
}
\end{lstlisting}

\textbf{Response (Error - File troppo grande):}
\begin{lstlisting}[language=json,style=jsonStyle]
{
  "status": "error",
  "message": "File too large"
}
\end{lstlisting}

%------------------------------------------------------------------------------
\subsection{PUT /books}

Aggiornamento di un libro esistente. L'\textbf{id} \`e obbligatorio.

\textbf{Auth:} JWT Required

\begin{nota}
\textbf{Ownership:} L'utente pu\`o modificare solo i propri libri. Il sistema verifica che \code{book.sellerId} corrisponda all'ID dell'utente autenticato.
\end{nota}

\textbf{Request Body:}
\begin{lstlisting}[language=json,style=jsonStyle]
{
  "id": 172,
  "price": 25.00,
  "available": false
}
\end{lstlisting}

\textbf{Response (Success):}
\begin{lstlisting}[language=json,style=jsonStyle]
{
  "status": "success",
  "message": "Book updated"
}
\end{lstlisting}

\textbf{Response (Error - Input non valido):}
\begin{lstlisting}[language=json,style=jsonStyle]
{
  "status": "error",
  "message": "Invalid input"
}
\end{lstlisting}

\textbf{Response (Error - Libro non trovato):}
\begin{lstlisting}[language=json,style=jsonStyle]
{
  "status": "error",
  "message": "Book not found"
}
\end{lstlisting}

\textbf{Response (Error - Non proprietario):}
\begin{lstlisting}[language=json,style=jsonStyle]
{
  "status": "error",
  "message": "Forbidden"
}
\end{lstlisting}

%------------------------------------------------------------------------------
\subsection{DELETE /books}

Rimozione di un libro.

\textbf{Auth:} JWT Required

\begin{nota}
\textbf{Ownership:} L'utente pu\`o eliminare solo i propri libri. Il sistema verifica che \code{book.sellerId} corrisponda all'ID dell'utente autenticato.
\end{nota}

\textbf{Request Body:} \code{\{"id": 78\}}

\textbf{Response (Success):}
\begin{lstlisting}[language=json,style=jsonStyle]
{
  "status": "success",
  "message": "Book deleted"
}
\end{lstlisting}

\textbf{Response (Error - Input non valido):}
\begin{lstlisting}[language=json,style=jsonStyle]
{
  "status": "error",
  "message": "Invalid input"
}
\end{lstlisting}

\textbf{Response (Error - Libro non trovato):}
\begin{lstlisting}[language=json,style=jsonStyle]
{
  "status": "error",
  "message": "Book not found"
}
\end{lstlisting}

\textbf{Response (Error - Non proprietario):}
\begin{lstlisting}[language=json,style=jsonStyle]
{
  "status": "error",
  "message": "Forbidden"
}
\end{lstlisting}

%------------------------------------------------------------------------------
\subsection{POST /purchase}

Effettua l'acquisto di un libro.

\textbf{Auth:} JWT Required

\textbf{Request Body:}
\begin{lstlisting}[language=json,style=jsonStyle]
{
  "bookId": 78
}
\end{lstlisting}

\textbf{Response (Success):}
\begin{lstlisting}[language=json,style=jsonStyle]
{
  "status": "success",
  "message": "Purchase completed successfully",
  "orderId": 501,
  "sellerEmail": "venditore@email.it"
}
\end{lstlisting}

\textbf{Response (Error - Libro non trovato):}
\begin{lstlisting}[language=json,style=jsonStyle]
{
  "status": "error",
  "message": "Book not found"
}
\end{lstlisting}

\textbf{Response (Error - Libro gi\`a venduto):}
\begin{lstlisting}[language=json,style=jsonStyle]
{
  "status": "error",
  "message": "Book already sold"
}
\end{lstlisting}

\textbf{Response (Error - Auto-acquisto):}
\begin{lstlisting}[language=json,style=jsonStyle]
{
  "status": "error",
  "message": "Cannot purchase your own book"
}
\end{lstlisting}

%------------------------------------------------------------------------------
\subsection{GET /purchases}

Recupera lo storico degli acquisti dell'utente autenticato.

\textbf{Auth:} JWT Required

\textbf{Response (Success):}
\begin{lstlisting}[language=json,style=jsonStyle]
{
  "status": "success",
  "data": [
    {
      "orderId": 501,
      "book": {
        "id": 78,
        "name": "Ingegneria del Software",
        "author": "Ian Sommerville",
        "price": 35.00
      },
      "sellerUsername": "mario_rossi",
      "purchaseDate": "2026-01-25T14:30:00Z"
    }
  ]
}
\end{lstlisting}

\textbf{Response (Error - Errore server):}
\begin{lstlisting}[language=json,style=jsonStyle]
{
  "status": "error",
  "message": "Server error"
}
\end{lstlisting}

%------------------------------------------------------------------------------
\subsection{GET /sales}

Recupera lo storico delle vendite dell'utente autenticato.

\textbf{Auth:} JWT Required

\textbf{Response (Success):}
\begin{lstlisting}[language=json,style=jsonStyle]
{
  "status": "success",
  "data": [
    {
      "orderId": 501,
      "book": {
        "id": 78,
        "name": "Ingegneria del Software",
        "author": "Ian Sommerville",
        "price": 35.00
      },
      "buyerUsername": "luigi_verdi",
      "saleDate": "2026-01-25T14:30:00Z"
    }
  ]
}
\end{lstlisting}

\textbf{Response (Error - Errore server):}
\begin{lstlisting}[language=json,style=jsonStyle]
{
  "status": "error",
  "message": "Server error"
}
\end{lstlisting}

%------------------------------------------------------------------------------
\subsection{GET /conversations}

Recupera la lista delle conversazioni dell'utente autenticato.

\textbf{Auth:} JWT Required

\textbf{Response (Success):}
\begin{lstlisting}[language=json,style=jsonStyle]
{
  "status": "success",
  "data": [
    {
      "userId": 5,
      "username": "luigi_verdi",
      "lastMessage": "Ok, ci vediamo domani",
      "lastMessageDate": "2026-01-25T14:30:00Z"
    }
  ]
}
\end{lstlisting}

\textbf{Response (Error - Errore server):}
\begin{lstlisting}[language=json,style=jsonStyle]
{
  "status": "error",
  "message": "Server error"
}
\end{lstlisting}

%------------------------------------------------------------------------------
\subsection{GET /messages?userId=\{id\}}

Recupera i messaggi scambiati con un utente specifico.

\textbf{Auth:} JWT Required

\textbf{Query Params:} \code{userId} (required)

\textbf{Response (Success):}
\begin{lstlisting}[language=json,style=jsonStyle]
{
  "status": "success",
  "data": [
    {
      "id": 1,
      "senderId": 5,
      "receiverId": 10,
      "content": "Ciao, il libro e' ancora disponibile?",
      "createdAt": "2026-01-25T14:00:00Z"
    }
  ]
}
\end{lstlisting}

\textbf{Response (Error - Utente non trovato):}
\begin{lstlisting}[language=json,style=jsonStyle]
{
  "status": "error",
  "message": "User not found"
}
\end{lstlisting}

%------------------------------------------------------------------------------
\subsection{POST /messages}

Invia un nuovo messaggio a un utente.

\textbf{Auth:} JWT Required

\textbf{Request Body:}
\begin{lstlisting}[language=json,style=jsonStyle]
{
  "receiverId": 5,
  "content": "Ciao, sono interessato al libro"
}
\end{lstlisting}

\textbf{Response (Success):}
\begin{lstlisting}[language=json,style=jsonStyle]
{
  "status": "success",
  "message": "Message sent",
  "messageId": 42
}
\end{lstlisting}

\textbf{Response (Error - Destinatario non trovato):}
\begin{lstlisting}[language=json,style=jsonStyle]
{
  "status": "error",
  "message": "User not found"
}
\end{lstlisting}

\textbf{Response (Error - Messaggio a se stesso):}
\begin{lstlisting}[language=json,style=jsonStyle]
{
  "status": "error",
  "message": "Cannot message yourself"
}
\end{lstlisting}

\textbf{Response (Error - Contenuto vuoto):}
\begin{lstlisting}[language=json,style=jsonStyle]
{
  "status": "error",
  "message": "Message content required"
}
\end{lstlisting}

%------------------------------------------------------------------------------

\begin{nota}
\textbf{Ricerca Avanzata (RF05):} La funzionalit\`a di ricerca per ISBN, corso e docente \`e implementata interamente lato Frontend mediante filtri JavaScript sui dati gi\`a recuperati tramite \code{GET /books}. Non sono previsti endpoint dedicati.
\end{nota}

%------------------------------------------------------------------------------
\section{Interfaccia Database}

\subsection{Entity Classes}

Oggetti PHP rappresentanti i dati scambiati tra i vari moduli. Il campo \code{id} \`e nullable per poter creare nuovi oggetti.

\begin{lstlisting}[style=phpStyle]
class User {
    public ?int $id;
    public string $username;
    public string $email;
    public string $password;
}

class Book {
    public ?int $id;
    public string $name;
    public string $author;
    public string $isbn;
    public string $imagePath;
    public string $teacher;
    public string $course;
    public float $price;
    public int $sellerId;
    public bool $available;
}

class Transaction {
    public ?int $id;
    public int $bookId;
    public int $buyerId;
    public int $sellerId;
    public DateTime $createdAt;
}

class Message {
    public ?int $id;
    public int $senderId;
    public int $receiverId;
    public string $content;
    public DateTime $createdAt;
}
\end{lstlisting}

\subsection{Metodi Gestione Utenti}

\begin{itemize}
    \item \code{registerUser(User \$user): bool} -- Esegue l'insert del nuovo utente.
    \item \code{verifyCredentials(string \$email, string \$password): ?int} -- Ritorna l'ID utente se le credenziali sono corrette (con verifica \code{password\_verify}), altrimenti \code{null}.
    \item \code{getUserById(int \$id): ?User} -- Ritorna l'utente con l'ID specificato, oppure \code{null} se non trovato.
\end{itemize}

\begin{nota}
Il metodo \code{getUserById} \`e stato aggiunto per supportare il recupero dell'email del venditore in \code{POST /purchase}.
\end{nota}

\subsection{Metodi Gestione Libri}

\begin{itemize}
    \item \code{getAllBooks(): array} -- Ritorna un array di oggetti \code{Book} disponibili.
    \item \code{insertBook(Book \$book): int} -- Inserisce un libro e ritorna l'ID generato.
    \item \code{updateBook(Book \$book): bool} -- Aggiorna i campi modificati.
    \item \code{deleteBook(int \$id): bool} -- Rimuove logicamente o fisicamente il record.
\end{itemize}

\subsection{Metodi Gestione Transazioni}

\begin{itemize}
    \item \code{placeOrder(int \$bookId, int \$buyerId): int} -- Crea un record nella tabella ordini, aggiorna \code{available = false} sul libro e ritorna l'ID ordine.
    \item \code{getPurchasesByBuyer(int \$buyerId): array} -- Ritorna lo storico acquisti dell'utente.
    \item \code{getSalesBySeller(int \$sellerId): array} -- Ritorna lo storico vendite dell'utente.
\end{itemize}

\subsection{Metodi Gestione Messaggi}

\begin{itemize}
    \item \code{getConversations(int \$userId): array} -- Ritorna la lista delle conversazioni con ultimo messaggio.
    \item \code{getMessages(int \$userId1, int \$userId2): array} -- Ritorna i messaggi tra due utenti ordinati cronologicamente.
    \item \code{sendMessage(Message \$message): int} -- Inserisce un messaggio e ritorna l'ID generato.
\end{itemize}
