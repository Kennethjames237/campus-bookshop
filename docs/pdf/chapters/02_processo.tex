% =============================================================================
% CHAPTER 2: PROCESSO DI SVILUPPO
% =============================================================================

\chapter{Processo di Sviluppo}

\section{Metodologia}

Il processo adottato \`e \textbf{Iterativo} ed \textbf{Incrementale}. Ogni Sprint trasforma un sottoinsieme di Requisiti in un incremento software funzionante, testato e documentato.

\subsection{A. Sprint Planning}

\begin{itemize}
    \item \textbf{Analisi dei Requisiti:} Revisione dei requisiti ed eventuale aggiornamento.
    \item \textbf{Selezione User Stories:} Spostamento delle user stories nello Sprintlog.
    \item \textbf{Analisi:} Le user stories selezionate vengono dettagliate in \code{docs/features/} tramite:
    \begin{itemize}
        \item Lo Use Case Diagram, con tabella Use Case per dettaglio ed eventuali Activity Diagram.
        \item I Criteri di Accettazione.
        \item La strategia per i Test di Integrazione manuali.
    \end{itemize}
    \item \textbf{Design Session:} Definizione dei contratti:
    \begin{itemize}
        \item Specifica JSON delle API.
        \item Specifica metodi interfaccia database.
    \end{itemize}
    \item \textbf{Task Assignment:} Suddivisione dei compiti, creazione dei branch \code{feature/} e creazione della issue relativa alla user story.
\end{itemize}

\subsection{B. Development Loop}

\begin{itemize}
    \item \textbf{Detailed Design:} Prima della codifica, ogni sviluppatore progetta il proprio componente nella cartella della feature:
    \begin{itemize}
        \item Diagrammi delle Classi.
        \item Diagrammi ER specifici.
    \end{itemize}
    \item \textbf{Coding \& Unit Testing:} Scrittura del codice sorgente e degli eventuali Unit Test.
    \item \textbf{Pull Request:} Lo sviluppatore apre una PR verso il ramo di integrazione (\code{develop}).
    \item \textbf{Merge:} Se la review \`e positiva e i test passano, il codice viene integrato in \code{develop}.
\end{itemize}

\subsection{C. Sprint Review}

\begin{itemize}
    \item \textbf{Integration Test:} Verifica completa del flusso basata sui \textbf{Criteri di Accettazione}.
\end{itemize}

\begin{figure}[H]
    \centering
    \includesvg[height=0.85\textheight]{../img/process-diagram.drawio}
    \caption{Activity Diagram del Processo di Sviluppo}
\end{figure}

\section{Organizzazione del Team}

\subsection{Database}
\begin{itemize}
    \item \textbf{Responsabile:} Andrea Belli
    \item \textbf{Responsabilit\`a:} gestione Database, gestione classe interfaccia Database.
\end{itemize}

\subsection{Backend}
\begin{itemize}
    \item \textbf{Responsabile:} Samuele Premori
    \item \textbf{Responsabilit\`a:} Logica business, esposizione API, test di unit\`a.
\end{itemize}

\subsection{Frontend}
\begin{itemize}
    \item \textbf{Responsabile:} Nna Minkousse Kenneth James
    \item \textbf{Responsabilit\`a:} Interfaccia utente, integrazione API, test manuali.
\end{itemize}

\section{Branch Strategy}

\subsection{Branch Structure}
\begin{itemize}
    \item \code{main}: codice stabile e testato
    \item \code{dev}: ramo di integrazione continua delle feature
    \item \code{feature/T\{ID\}}: ramo riguardante la task specifica, singolo per sviluppatore.
\end{itemize}

\subsection{Regole}
\begin{itemize}
    \item Vietato il push diretto su \code{main}.
    \item Il push diretto su \code{dev} avviene solo per modifiche della documentazione o modifiche tra gli sprint approvate in maniera unanime.
    \item Ogni ramo \code{feature} si chiude con un pull request su \code{dev}.
    \item Ogni pull request da \code{dev} a \code{main} deve contenere la documentazione aggiornata e il tag relativo alla versione.
\end{itemize}

\section{Testing}

La strategia di testing \`e divisa in:

\begin{itemize}
    \item \textbf{Unit Testing:}
    \begin{itemize}
        \item Applicato alle classi del Backend.
        \item Obbligatorio passare i test prima della PR.
        \item Framework: PHPUnit.
    \end{itemize}
    
    \item \textbf{Integration Testing:}
    \begin{itemize}
        \item Eseguito manualmente seguendo i file di collaudo compilati a inizio sprint.
    \end{itemize}
\end{itemize}
